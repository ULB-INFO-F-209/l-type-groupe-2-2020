\documentclass[a4paper,12pt]{article}

\usepackage[utf8]{inputenc}

\usepackage[parfill]{parskip}

\usepackage[T1]{fontenc}
\usepackage[french]{babel}
\usepackage{array,multirow,makecell}
\usepackage{longtable}
\usepackage{setspace}
\usepackage{makecell}
\setcellgapes{1pt}
\makegapedcells
\usepackage[table]{xcolor}
\renewcommand*{\emph}[1]{\textcolor{green}{#1}}
\newcolumntype{R}[1]{>{\raggedleft\arraybackslash }b{#1}}
\newcolumntype{L}[1]{>{\raggedright\arraybackslash }b{#1}}
\newcolumntype{C}[1]{>{\centering\arraybackslash }b{#1}}
\usepackage{amsfonts}
\usepackage{fullpage}
\usepackage{graphicx}
\usepackage{float}
\usepackage{geometry}
\usepackage{amsmath}
\usepackage{amssymb}
\usepackage{xspace}
\usepackage{epstopdf}
\usepackage{tabularx}

% -----------------------------------------------------
\begin{document}

    \begin{titlepage}

        \begin{center}

            \title{\Huge INFO-F209 Projets d’informatique 2 \\
            [1 cm] Software Requirements Document \\ 
            L-type\\[2 cm]}
            \author{Aïssa ABDOUL-AZIZ \\[0,2 cm] Kokou ADEGNON \\[0,2 cm] Jeremy BARBER \\[0,2 cm] Helin DEMIREL \\[0,2 cm] Camelia ELKENZE \\[0,2 cm] Alexandre KINSOEN \\[0,2 cm] Salim Latoundji \\[0,2 cm] Mario MASSIMETTI \\[0,2 cm] Martin VANNESTE \\ [2 cm]}
            \date{Décembre 2020}

            %\includegraphics[width = 40mm]{ulb.png}

        \end{center}

    \end{titlepage}

    \maketitle

\newpage

\tableofcontents

\newpage

% -----------------------------------------------------

\section{Introduction}

L’objectif de ce projet consiste en la réalisation d'un jeu d'action de style shoot 'em up en multijoueur. Dans ce jeu, un ou deux joueurs doivent parcourir plusieurs niveaux en détruisant les ennemis qui se présentent devant eux, tout en esquivant les tirs provoqués par ces derniers. Les vaisseaux dirigés par les joueurs peuvent récupérer des bonus d’armement lâchés par leurs nombreux adversaires pour mieux les éliminer. Le but étant de terminer tous les niveaux sans perdre toute ses vies. En effet, un joueur en possède un nombre déterminé. Si le projectile d'un joueur touche un ennemi, son score est augmenté. Et à la fin d'une partie, le score de chaque utilisateur est mis à jour dans son compte.

En dehors du jeu, un utilisateur a la capacité de gérer sa liste d'amis, de discuter avec d'autres utilisateurs et de consulter le classement général des joueurs.

Le jeu ne sera exécutable que sous le système d'exploitation Linux.

\subsection{Historique}
\begin{tabularx}{15cm}{|c|c|X|}
	\hline
		Dates & Sujets & Noms \\
	\hline
		13/11/20 & UseCase Utilisateur & Camelia,Jeremy,Salim \\
	\hline
		22/11/20 & Annexe & Jeremy ,Camelia \\
	\hline
		10/12/20 & Besoins utilisateur & Kokou, Camelia, Helin, Aissa\\
	\hline
		11/12/20 & Version finale diagramme de classe  & Jeremy,Martin,Salim,\\
	\hline
		14/12/20 & Version final des diagrammes de sequence & Helin, Aissa, Martin, Kokou,Camelia, 
		Mario, Alexandre, Martin, Jeremy, Salim\\
	\hline
		15/12/20 & Introduction du SRD & Helin, Aissa,Mario\\
	\hline
		15/12/20 & Besoins système partie serveur & Alexandre, Jeremy, Martin, Camelia,Aissa,Helin\\
	\hline
		15/12/20 & Besoins système partie client & Kokou, Mario, Salim\\
	\hline
		16/12/20 & Description du diagramme de classe & Helin,Mario,Salim,Aissa,
		Alexandre,Jeremy,Martin\\
	\hline

\end{tabularx}

\section{Besoins utilisateur}

\subsection{Besoins fonctionnels}

% ajoute de l'image

\begin{figure}[h!]
\centering
\includegraphics[width=15cm]{UserUseCase}
\caption{Diagramme de use case coté utilisateur}
\label{fig:UerUseCase}
\end{figure}

\subsubsection{Connexion}
En lançant le programme, l'utilisateur est invité à s'inscrire ou se connecter.

A l'inscription, un pseudonyme unique et un mot de passe lui seront demandé tant que le pseudo entré est déjà pris par quelqu'un d'autre.

Dans le cas de la connexion, on invite l'utilisateur à saisir son pseudo et son mot de passe tant que le pseudo est inexistant ou que le mot de passe ne correspond pas. Après s’être connecté l'utilisateur accède au menu principal.

Dans les deux cas, une option de retour à la page d'accueil sera disponible.

\subsubsection{Menu principal}

- Discuter:

La liste des chats existant est présentée à l'utilisateur, il a le choix de discuter dans un chat existant ou d'en créer un nouveau avec un utilisateur. 

L'utilisateur peut aussi voir si il a de nouveaux messages non-lus.\\

- Consulter amis:

La liste des amis de l'utilisateur est affichée lorsque cette option est sélectionnée. Il peut supprimer un ami, en ajouter un nouveau et aussi voir ses demandes d'amis. 

La suppression d'un ami X par l'utilisateur Y implique la suppression de Y dans la liste d'amis de X. 

L'ajout d'un ami est en fait une invitation. La personne sera l'ami de l'utilisateur seulement si elle accepte sa demande. \\

- Consulter le classement:

Le classement affiche le score de tout les utilisateurs. Si il le souhaite, l'utilisateur peut afficher seulement le classement de score de ses amis. \\

-Envoyer une demande de jeu:

L’utilisateur a la possibilité d’inviter un autre utilisateur, avec un pseudonyme valide, à rejoindre une partie. Si ce dernier est connecté et qu’il n’est pas déjà entrain de jouer, il a possibilité d’accepter ou de refuser la demande. S’il accepte la connexion entre les deux joueurs sera établie, s’il refuse l’hôte ne sera pas averti. \\

-Consulter les demandes de jeu:

Lors d’une consultation de la liste d’invitations, la possibilité est laissée à l’utilisateur d’accepter ou de décliner l’invitation dans le cas ou une invitation est présente dans la liste.Ainsi, accepter une invitation transfère l’utilisateur dans le lobby de l’hôte si ce lobby n’a pas atteint son nombre maximum de joueurs. L’invité ne peut pas modifier les paramètres que le premier joueur aura choisi.A l’inverse refuser une invitation n’avertit pas l’hôte. \\

-Lancer une partie:

Le lancement d’une partie se fait lorsque l’utilisateur crée une partie en gardant les paramètres par défauts ou en les redéfinissants.

- Paramètres:

Dans les paramètres, l'utilisateur peut consulter les règles du jeu avec le bouton "Help". Il peut également configurer ses préférences audio, visuel, ainsi qu'accéder à son profil. Dans son profil, il peut consulter ses informations de compte et s'il le souhaite, changer son mot de passe.

\subsubsection{Création de partie}
La création d'une partie est une option qui envoie l'utilisateur vers une fenêtre de personnalisation permettant de modifier les paramètres du jeu.
Cette fenêtre contient déjà des paramètres par défauts.
S'il le souhaite l'utilisateur peut définir :

-Le nombre de joueur: 

Un joueur a la possibilité de choisir entre un ou deux participants.Dans le cas où le nombre de participants équivaut à deux il aura la possibilité d'entrer un pseudo valide et la partie ne se lancera que si le joueur invité accepte la demande de jeu. \\

-La difficulté de la partie: 

Chaque partie est composée de plusieurs niveaux de difficulté qui augmente progressivement.  \\

-La probabilité d'apparition des bonus:

L'utilisateur a le choix de gérer la probabilité qu'aura un ennemi de lâcher un bonus après sa destruction. \\

-Le tir allié: 

La possibilité d'activer le tir allié ne peut être accordé que dans le cas où le nombre de joueur est supérieur à un.
Dans ce premier scénario le joueur a le droit de choisir s'il souhaite que les projectiles de l'invité soit inoffensifs ou non. Cette option vaut pour les deux joueurs. \\

-Le nombre de vies: 

Le choix du nombre de vies est décidé par l'utilisateur. \\

\subsubsection{Jeu}

Un joueur commence sa partie avec un nombre de vies prédéterminé et un score nul. 
Il contrôle un vaisseau avec lequel il peut tirer des projectiles vers des ennemis, ce qui augmentera son score selon la quantité de dommage infligée. Un vaisseau peut se déplacer dans toutes les directions. Celui d'un joueur peut également attraper des bonus qui lui donneront des améliorations d'armes. Ces bonus sont lâchés lors de la destruction d'ennemis. Le vaisseau joueur peut aussi subir des dégâts, qui vont réduire son niveau de vie. Si ce niveau est nul, le vaisseau est détruit et le nombre de vies est décrémenté. Un nouveau vaisseau est attribué au joueur s'il possède encore des vies. Sinon, la partie est terminée pour lui.

Un jeu est composé de plusieurs niveaux. Un niveau varie selon le nombre d'ennemis, de leur résistance, de la puissance de leur projectiles. Mais elle dépend également du nombre d'obstacles.

\newpage

\section{Besoins système}
\subsection{Besoins fonctionnels}

\begin{figure}[h!]
\centering
\includegraphics[width=16cm]{newSystemClassDiagram.jpg}
\caption{Diagramme de classe du système}
\label{fig:UerUseCase}
\end{figure}

\subsubsection{Serveur}
Le serveur est la classe principale du système. C'est la classe qui va gérer la majorité des interactions du programme. Il va servir d'intermédiaire entre le client et toutes les fonctionnalités du jeu. Pour le bon fonctionnement du jeu, cette classe devra toujours être active.\\
Parmi les fonctionnalités principales du serveur, on peut trouver:\\
- la connexion entre client et serveur
- l'interaction entre les utilisateurs\\
- la gestion des données personnelles\\
- la gestion de toutes les parties en cours
\subsubsection{Database}
La classe Database porte bien son nom puisqu'elle sauvegardera toutes les informations persistantes et nécessaires au bon fonctionnement du jeu.\\
Elle se chargera de:\\
- préserver et manipuler les informations personnelles relatives à chaque joueur\\
- conserver les historiques de discussions de tous les joueurs\\
- la vérification de l'existence des données
\subsubsection{Lobby}
Le Lobby va servir de salle d'attente pour un ou deux joueurs avant de lancer une partie commune. L'hôte d'un lobby va pouvoir décider de son accessibilité (privé/publique).

\subsubsection{Client}
La classe Client permet à l'utilisateur de communiquer avec le serveur à travers diverses actions possibles affichées grâce au menu.
Toutes les fonctionnalités décrites dans la section 2.1 sont possibles :
\begin{itemize}
    \item Se connecter
    \item S'inscrire
    \item rejoindre un lobby
    \item crée une partie
    \item discuter avec d'autres utilisateurs
    \item consulter le classement
    \item ajouter ou supprimer un ami
\end{itemize}

Les actions citées précédemment ne sont que si le client possède une connexion avec le serveur. Les entrées du client capturer par la classe "Input".\\
La classe Input est connecter au menu en cours d'affichage, ce qui permet de transmettre les entrées au client qui lui, les envoie au serveurs.\\
La classe abstrait "Menu" a pour but de représenter une interface au niveau graphique ou terminale. Les classes héritières de la classe abstraite devrons afficher le menu qui leur correspond et vérifier a la demande de la classe "Input" la validité d'une entrée.\\
La classe "Display" permet de gérer l'affichage des objets du jeu pour le client. Cette classe possédera plus de méthodes dans la partie graphique du projet. 


\subsubsection{Gestion des comptes}
Un objet Account contiendra toutes les informations relatives à un utilisateur (nom, pseudo,...). Ces informations modulables seront stockées sur la base de données.
 Les différentes demandes d'accès à la base de données sont traitées par le serveur. En effet, celui-ci permet la liaison entre le compte et l'utilisateur.

- Accès:

Lors de la création d'un compte, la disponibilité du pseudo est vérifiée par le serveur. Si celui-ci n'est pas trouvé, un nouveau compte est bien créé et ajouté dans la base de données.

Lors de la connexion, c'est encore le serveur qui vérifie que le pseudo et le mot de passe saisis correspondent à un compte existant.

- Contenu d'un compte:

Chaque compte possède des informations sur l'utilisateur auquel il appartient, notamment ses identifiants(pseudo, mot de passe). Son score est aussi présent pour qu'il apparaisse dans le classement général des joueurs. Il a aussi une liste d'amis qu'il peut consulter à tout moment.

\subsubsection{Gestion d'une partie}

Tout client connecté sera placé dans un lobby privé par défaut. Lorsqu'il voudra lancer une partie,
il aura la possibilité de choisir les paramètres de celle-ci comme expliqué plus haut (cf 2.1.3). Le joueur peut aussi changer le statut du lobby(public/privé). Si le nombre de joueurs choisi est 2, l'hôte doit obligatoirement inviter un autre utilisateur connecté dans le cas où le lobby est privé.
Le serveur vérifie que la personne invitée existe dans la base de donnée, qu'elle est connectée et qu'elle n'est pas dans une partie en cours.
S'il est public, tout utilisateur connecté pourra rejoindre le lobby qui devient ainsi privé.
La partie ne peut pas commencer tant que le nombre de joueurs spécifiés ne correspond pas au nombre de personnes présentes dans le lobby.

\subsubsection{Gestion des amis}

- Ajout: Lorsque l'utilisateur voudra ajouter un ami, le serveur fera des vérifications et enverra la demande, si elle est valide, à la personne concernée. Si la demande est acceptée, la liste d'amis des deux utilisateurs est mise à jour par le serveur.

- Suppression: Le serveur fait une recherche de l'ami à supprimer et efface dans la liste d'amis de celui-ci, l'utilisateur ayant fait la demande de suppression. Ensuite, l'ami est effacé dans la liste de l'utilisateur.

\subsubsection{Chat}

Les fichiers de chat entre deux utilisateurs sont stockés dans la base de données. Ces fichiers ont comme nom "pseudo1\_pseudo2.txt" où pseudo1 est plus petit que pseudo2. Chaque ligne du fichier a comme format "pseudo: message".

L'utilisateur peut discuter avec n'importe quel autre utilisateur. Lorsqu'il veut ouvrir un chat, le serveur vérifie qu'il n'existe pas déjà un fichier chat partagé par ces deux personnes. Si oui, son contenu est affiché, sinon, un nouveau fichier est créé et stocké dans la base de données. Chaque nouveau message entraîne l'écriture dans le fichier et une incrémentation du nombre de messages non-lus du destinataire.

\subsubsection{Classement}

Après chaque partie, le serveur met à jour le score des joueurs si ils ont battu leur record de meilleur score.

\subsection{Besoins non fonctionnels}

Le lancement du programme nécessite un environnement Linux. Par souci de sécurité, toutes les requêtes d'un client doivent passer par le serveur.

\newpage
\section{Annexes}
\subsection{ Description du diagramme use case utilisateur}

\begin{center}
\begin{longtable}{|p{1,5cm}||p{3,5cm}|p{3,5cm}|p{3,5cm}|p{3,5cm}|}
\hline
\rowcolor{green}
USE CASE   &\center{Pré-conditions}   & \hfill Post-conditions \hfill\null & Cas Général & Cas exceptionnels\\
\hline
\hline
\textbf{Sign in}      & L'utilisateur doit être enregistré dans la base de données  & L'utilisateur est connecté à sa base de données et le menu principal est affiché & L'utilisateur déjà enregistré se connecte à son compte en tapant son nom d’utilisateur et mot de passe. Le système vérifie que les données soient correctes et donne accès au compte du client.  & Si l’identifiant ou le mot de passe sont incorrectes, le système affiche un message d'erreur a l'utilisateur. \\
\hline
\hline
\textbf{Sign up}     & L'utilisateur n'est pas présent dans la base de données   & Ajout d'un compte dans la base de données et affichage du menu principal & L'utilisateur crée un compte en introduisant un pseudo et un mdp  & Si les données entrées ne respectent pas le format requis ou que le nom d'utilisateur est déjà utilisé, un message d'erreur est affiché et l'utilisateur doit recommencer l'action jusqu'à ce que ce soit valide \\

\hline
\hline
\textbf{Create game}    & L'utilisateur doit être enregistré dans la base de données  & Possibilité de sauvegarder les options requises  & L'utilisateur peut lancer une partie après avoir rempli les conditions minimales  & Néant \\
\hline
\hline
\textbf{Invite player} & L'utilisateur doit être enregistré dans la base de données   & Etablissement de la connexion via le serveur entre l’hôte et l’invité. & Inviter un joueur avec son pseudo & Invitation via pseudo qui n’existe pas. \\
\hline
\hline
\textbf{Player request}  & Recevoir une invitation via le serveur.
L'utilisateur doit être enregistré dans la base de données   & Etablissement de la connexion via le serveur entre l’hôte et l’invité.  & Accepter/ Refuser une invitation de partie  & La connexion échouera si :
Accepter une invitation dont l'hôte n’est plus connecté.
Rejoindre un salon complet. \\
\hline
\hline
\textbf{Check Leaderboard}  & L'utilisateur doit être enregistré dans la base de données   & \hfill Néant  \hfill\null &Le joueur peut consulter le classement des meilleurs scores en envoyant une requête au serveur qui va lui renvoyer les informations  & Néant \\
\hline
\hline
\textbf{View friend list }   & L'utilisateur doit être enregistré dans la base de données   & Néant  & Consultation de liste d'ami dans la base de données. & Néant \\
\hline
\hline
\textbf{Chat}     & L'utilisateur doit être enregistré dans la base de données   & Le serveur effectue les liaisons entre les utilisateurs  & Envoyer et recevoir des messages avec n'importe quel pseudo.  & Néant \\
\hline
\hline
\textbf{Add friend}    & L'utilisateur doit être enregistré dans la base de données   & Si invitation acceptée, ajout d'amis dans la base de donnée (bidirectionnel).  & Entrer le pseudo d’un ami. Le système va rechercher dans la base de données si le pseudo existe et l’ajouter.  & Ajouter un ami qui n’existe pas (affiche une erreur).\\
\hline
\hline
\textbf{Delete friend}    & L'utilisateur doit être enregistré dans la base de données et avoir au moins un ami.   & Suppression d’amis(de la liste d’amis) de la base de données.  & Entrer le pseudo d’un ami. Le système va rechercher dans la base de données si le pseudo existe et le supprimer.  & Supprimer un ami qui n’existe pas (affiche une erreur). \\
\hline
\hline
\textbf{Settings}     & L'utilisateur doit être enregistré dans la base de données   & Mise à jour des changements  & L'utilisateur modifie des parametres de son profil de l'affichage ou du son  & Néant\\
\hline
\hline
\textbf{Ship’s controls}  & Charger une partie/créer une partie.   & Actualisation de l’état de jeu.  & Bouger, tirer recevoir des bonus.  & Néant\\
\hline
\hline
\textbf{Leave party}      & Être en train de jouer   & Retour au menu principal  & Le joueur arrête la partie en cours.  & Néant\\
\hline
\hline
\textbf{Help}      & Etre connecté   & Néant  & Les règles principales du jeu sont affichées au joueur & Néant\\
\hline
\hline
\textbf{Leave game}      & Etre connecté   & Fermeture du jeu  & L'utilisateur est connecté et veut quitter le jeu  & L'utilisateur est connecté et force sa sortie du jeu (CTRL+C, ...)\\
\hline
\end{longtable}
\end{center}


\end{document}
