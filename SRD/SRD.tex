\documentclass[a4paper,12pt]{article}

\usepackage[utf8]{inputenc}

\usepackage[parfill]{parskip}

\usepackage[T1]{fontenc}
\usepackage[french]{babel}
\usepackage{array,multirow,makecell}
\usepackage{longtable}
\usepackage{setspace}
\usepackage{makecell}
\setcellgapes{1pt}
\makegapedcells
\usepackage[table]{xcolor}
\renewcommand*{\emph}[1]{\textcolor{green}{#1}}
\newcolumntype{R}[1]{>{\raggedleft\arraybackslash }b{#1}}
\newcolumntype{L}[1]{>{\raggedright\arraybackslash }b{#1}}
\newcolumntype{C}[1]{>{\centering\arraybackslash }b{#1}}
\usepackage{amsfonts}
\usepackage{fullpage}
\usepackage{graphicx}
\usepackage{float}
\usepackage{geometry}
\usepackage{amsmath}
\usepackage{amssymb}
\usepackage{xspace}
\usepackage{epstopdf}
\usepackage{tabularx}
\usepackage{hyperref}

% -----------------------------------------------------
\begin{document}

    \begin{titlepage}

        \begin{center}
        
            \begin{figure}[H]
              \begin{minipage}[c]{.46\linewidth}
                    \centering
                    \includegraphics[scale = 0.3]{images/sceau_ulb.png}
                \end{minipage}
                \hfill%
                \begin{minipage}[c]{.46\linewidth}
                    \centering
                    \includegraphics[scale=0.5]{images/logo_ulb.png}
                \end{minipage}
            \end{figure}
        
            {\\[2 cm] \Huge\\INFO-F209 - Projets d'informatique 2 \\ Software Requirements Document \\ [1 cm]
            L-type\\[2 cm]}
            {ABDOUL-AZIZ Aïssa \\[0,2 cm] ADEGNON Kokou  \\[0,2 cm] BARBER Jeremy \\[0,2 cm] DEMIREL Helin \\[0,2 cm] ELKENZE Camelia  \\[0,2 cm] KINSOEN Alexandre  \\[0,2 cm] LATOUNDJI Salim  \\[0,2 cm] MASSIMETTI Mario  \\[0,2 cm] VANNESTE Martin  \\ [2 cm] Mars 2021}
        \end{center}
    \end{titlepage}

\newpage

\tableofcontents

\newpage

% ----------------------------------------------------

\section{Introduction}

L’objectif de ce projet consiste en la réalisation d'un jeu d'action de style shoot 'em up en multijoueur. Dans ce jeu, un ou deux joueurs doivent parcourir plusieurs niveaux en détruisant les ennemis qui se présentent devant eux, tout en esquivant les tirs provoqués par ces derniers. Les vaisseaux dirigés par les joueurs peuvent récupérer des bonus d’armement lâchés par leurs nombreux adversaires, pour mieux les éliminer. L'objectif des joueurs étant de terminer tous les niveaux sans que leur compteur de vies ne se retrouve à 0. En effet, un joueur en possède un nombre déterminé. Si un projectile d'un joueur touche un ennemi, son score est augmenté. À la fin d'une partie, le score de chaque utilisateur est mis à jour si celui-ci est meilleur que son score actuel.

En dehors du jeu, un utilisateur a la capacité de gérer sa liste d'amis et de consulter le classement général des joueurs. De plus, il a la possibilité de créer, tester et évaluer des niveaux personnalisés de tous les joueurs enregistrés.

Le jeu ne sera exécutable que sous le système d'exploitation Linux.

\subsection{Historique}
\begin{tabularx}{16cm}{|c|c|X|}
	\hline
		Dates & Sujets & Noms \\
	\hline
		13/11/20 & Use case Utilisateur & Camelia, Jeremy, Salim \\
	\hline
		22/11/20 & Annexe & Jeremy, Camelia \\
	\hline
		10/12/20 & Besoins utilisateur & Kokou, Camelia, Helin, Aissa\\
	\hline
		11/12/20 & Version finale diagramme de classe  & Jeremy, Martin, Salim, Alexandre\\
	\hline
		14/12/20 & Version final des diagrammes de séquence & Helin, Aissa, Martin, Kokou, Camelia, 
		Mario, Alexandre, Jeremy, Salim\\
	\hline
		15/12/20 & Introduction du SRD & Helin, Aissa, Mario\\
	\hline
		15/12/20 & Besoins système partie serveur & Alexandre, Jeremy, Martin, Camelia, Aissa, Helin\\
	\hline
		15/12/20 & Besoins système partie client & Kokou, Mario, Salim\\
	\hline
		16/12/20 & Description du diagramme de classe & Helin, Mario, Salim, Aissa,
		Alexandre, Jeremy, Martin\\
	\hline
\end{tabularx}


\begin{tabularx}{16cm}{|c|c|X|}
	\hline
	    03/02/21 & Modifications concernant le SRD & Helin, Mario, Salim, Aissa,
		Alexandre, Jeremy, Martin, Kokou, Camelia\\
	\hline
	    15/02/21 & Modification diagrammes de classe jeu & Alexandre, Mario, Salim\\
	\hline
    	26/02/21 & Modification diagrammes de classe
	    jeu, client et serveur & Aissa, Jeremy, Helin, Camelia, Martin\\
	\hline
	    10/03/21 & Mise à jour du tableau use case + diagramme use case &  Aissa, Helin, Kokou, Salim\\
	\hline
	    10/03/21 & Dernières vérifications & Kokou\\
	\hline
	    01/04/21 & Explication de l'éditeur de niveaux & Aissa, Helin, Kokou, Salim\\
	\hline
	    5/04/21 & Mise à jour du déroulement du jeu graphique & Alexandre, Camelia, Jeremy, Mario, Martin\\
	\hline
	    20/04/21 & Ajout d'images + explication de jeu graphique & Alexandre, Camelia, Jeremy, Mario, Martin\\
	\hline
		 20/04/21 & Modification des besoins systèmes &  Aissa, Helin, Kokou, Salim\\
	\hline
\end{tabularx}

\newpage

\section{Besoins utilisateur}

\subsection{Besoins fonctionnels}

% ajoute de l'image

\begin{figure}[h!]
\centering
\includegraphics[width=15cm]{images/UserUseCase.png}
\caption{Diagramme de use case côté utilisateur}
\label{fig:UserUseCase}
\end{figure}

\subsubsection{Connexion}
En lançant le programme, l'utilisateur est invité à s'inscrire ou se connecter.

A l'inscription, un pseudonyme unique et un mot de passe lui sont demandés tant que le pseudo entré est déjà pris par quelqu'un d'autre.

Dans le cas de la connexion, on invite l'utilisateur à saisir son pseudo et son mot de passe tant que le pseudo est inexistant ou que le mot de passe ne correspond pas. Après s’être connecté, l'utilisateur accède au menu principal.

Dans les deux cas, une option de retour à la page d'accueil sera disponible.

\subsubsection{Menu principal}

\\- Gestion des amis:

Dans ce menu, l'utilisateur peut consulter sa liste d'amis et voir le score de ces derniers. Il a la possibilité d'envoyer 
des demandes d'amitié, en accepter ou en refuser. L'utilisateur peut également supprimer des personnes de sa liste d'amis.\\

- Consulter le classement:

Le classement affiche tous les utilisateurs de l'application ainsi que leur score. La liste est triée du joueur ayant le meilleur score au joueur ayant le moins bon.  
Cette option permet au client de prendre connaissance des pseudonymes des autres pour envoyer des demandes d'amis.\\

- Lancer une partie:

Lorsque l'utilisateur souhaite créer une partie, il se trouve dans un lobby (voir point 2.1.3). Il peut y modifier les paramètres par défaut du jeu et inviter un second joueur.
La partie est lancée quand l'hôte appuie sur play.\\

- Profile:

Dans le menu profile, l'utilisateur peut consulter ses informations de compte s'il le souhaite. A savoir, son pseudo et son score.


\begin{figure}[hbtp]
    \centering
    \includegraphics[scale=0.5]{images/menuGui.png}
    \caption{Menu version graphique }
    
    \includegraphics[scale=0.5]{images/menuTerm.png}
    \caption{Menu version terminal}
\end{figure}

\newpage

\subsubsection{Création de partie}
La création d'une partie est une option qui envoie l'utilisateur dans un lobby permettant de modifier les paramètres du jeu.
Cette fenêtre contient déjà des paramètres par défaut.

S'il le souhaite, l'utilisateur peut modifier les options suivantes:

- Le nombre de joueurs: 

L'utilisateur a la possibilité de choisir entre un ou deux participants. Dans le cas où le nombre de participants équivaut à deux, la deuxième personne est invitée à se connecter.\\

- La difficulté de la partie:

 L'utilisateur doit choisir le niveau de difficulté  de sa partie. La difficulté de chaque niveau sera adaptée en fonction de ce choix, mais elle augmente progressivement au fur et à mesure, après chaque victoire de niveau. \\

- Le tir allié: 

La possibilité d'activer le tir allié ne peut être accordée que dans le cas où le nombre de joueurs est supérieur à un.
Le joueur aura donc le droit de choisir s'il souhaite que les projectiles de l'invité soient inoffensifs ou non et inversement. Cette option active aussi la collision entre les joueurs. \\

- Le nombre de vies: 

Le choix du nombre de vies est limité à 3 pour l'utilisateur.  Il peut donc seulement la diminuer.\\

- Bonus :
Il existe différentes sortes de bonus:
\begin{itemize}
  \item DamageUp : augmente les dégâts des projectiles
  \item Minigun : tir automatique à cadence élevée
  \item TripleShot : tir de 3 projectiles en même temps
  \item LifeSteal : chaque élimination redonne de la vie
\end{itemize}

Le joueur peut choisir la probabilité d'apparition des bonus (bonus apparaissant lorsqu'il tue un ennemi).
Ainsi, il peut rendre le jeu plus ou moins difficile. 

\subsubsection{Jeu}

- Partie en cours:\\ \\
Le joueur commence chaque partie avec une à trois vies constituée(s) de 100 points de vie (sauf en cas de partie personnalisée) et
contrôle un vaisseau avec lequel il peut tirer des projectiles vers des ennemis, ce qui 
augmentera son score à chaque tir atteignant sa cible. Sa quantité de points de vie est réduite
lorsqu’il subit des dégâts. S’il n’a plus de points de vie, le joueur perd une vie. Lorsque le 
joueur n’a plus de vie, son vaisseau est détruit et si plus aucun joueur n’a de vie, la partie se 
termine. Elle peut également s’arrêter en cas de déconnexion d’un utilisateur, ou lorsque celui-ci a éliminé tous les ennemis.\\

\begin{figure}
\centering
\includegraphics[scale=0.5]{images/launch_game_sequence_diagram.jpg}
\caption{Diagramme de lancement d'une partie}
\end{figure}


- Niveau (par défaut):\\ \\
La partie est divisée en 4 niveaux de difficulté croissante. Au niveau 2 et 4, le(s) joueur(s) se retrouvera/ont face à un boss dont la difficulté dépendra du niveau. En passant au niveau suivant, le nombre, la santé et 
les dégâts des ennemis accroissent.\\

- Niveau (personnalisé):\\ \\
Les parties personnalisées sont constituées d'un seul niveau complètement modulable d'une durée maximum de 200 secondes. Durant cette partie, l'utilisateur fera face aux ennemis positionnés selon ses désirs et devra affronter un boss s'il en a décidé ainsi.\\

- Entités à l’écran:\\ \\
Les vaisseaux, les bonus, les projectiles et les obstacles
sont toutes des entités apparaissant et pouvant se déplacer sur l’écran.\\

\begin{itemize}
    \item[$\bullet$ Ship:] Il existe 4 types de ships : les PlayerShip, les EnemyShip (type 1 et 2) et les Boss. Tous ces vaisseaux peuvent subir et infliger des dégâts. 
    \begin{itemize}
        \item Les PlayerShip sont les vaisseaux alliés, ce sont ceux que les joueurs contrôlent. Ils peuvent se déplacer dans toutes les directions (verticales, horizontales, diagonales).
        \item Les EnemyShip peuvent lâcher un bonus en se détruisant, la probabilité de lâcher ce bonus est déterminée par l’utilisateur avant de lancer la partie. Le premier type peut uniquement se déplacer de manière verticale, alors que le second uniquement de manière horizontale.
        \item Le Boss est l’ennemi final ̀à affronter pour finir les niveaux. Le premier boss est un vaisseau beaucoup plus volumineux et résistant que les EnemyShip. Il a la capacité de se déplacer horizontalement de gauche à droite et de droite à gauche. Il peut également tirer 2 projectiles à la fois. Le second Boss détient les mêmes caractéristiques, tout en ayant la capacité de tirer 4 projectiles à la fois au lieu de 2.\\
    \end{itemize}

    \item[$\bullet$ Projectiles:]Les projectiles sont créés lorsque les vaisseaux tirent. Ils peuvent sortir de l’écran, s’annuler en rencontrant d’autres projectiles ou causer des dégâts en atteignant leur cible.\\

    \item[$\bullet$ Bonus:]Les bonus sont de plusieurs types et peuvent rapporter des améliorations d’armes et de santé au joueur qui réussit à l'attraper (voir point 2.3.1 -Bonus).
\end{itemize}

\subsubsection{Éditeur de niveau}

Création d'un niveau : 

L'éditeur de niveau est une option uniquement disponible dans la version graphique du programme. Il permet aux utilisateurs de créer un niveau personnalisé.

Le créateur pourra ajouter autant de vaisseaux ennemis qu'il désire et les ajuster à sa convenance.
En effet, il pourra décider de leur moment d'apparition et choisir leur type. Ce type aura un impact sur l'image du vaisseau et sur la direction de son mouvement (vertical ou horizontal). Il devra aussi décider du niveau de vie de l'ennemi, de la quantité de dégât que ce dernier pourra infliger à l'aide de ses projectiles ainsi que du bonus que ce vaisseau devra lâcher si la probabilité le permet. Il est laissé libre choix à l'utilisateur de choisir la position d'apparition des vaisseaux avec un "glisser-déposer" (drag and drop) directement sur l'écran de jeu. 

Le créateur peut aussi mettre des obstacles personnalisés de la même manière que les ennemis hormis le choix du type et du bonus qui ne sont pas disponibles. 

L'utilisateur peut en plus décider de la vitesse du jeu, activer ou non l'apparition d'un boss, choisir le niveau de vie ainsi que la quantité de dégâts des vaisseaux joueurs. 

Jouer un niveau: 

Les utilisateurs pourront jouer à leurs propres niveaux ou aux niveaux des autres après avoir choisi les paramètres usuels du lobby. Ils pourront interagir en votant pour les niveaux qu'ils ont aimé autant de fois qu'ils le souhaitent et permettre à ceux-ci de monter en haut du classement. 

\newpage



\newpage

\section{Besoins système}
\subsection{Besoins fonctionnels}

\begin{figure}[h!]
\centering
\includegraphics[width=16cm]{images/newSystemClassDiagram.png}
\caption{Diagramme de classe du système}
\label{fig:UerUseCase}
\end{figure}

\subsubsection{Serveur}
Le serveur est la classe principale du système. Il va gérer la majorité des interactions du programme et va servir d'intermédiaire entre le client et toutes les fonctionnalités de l'application. Pour le bon fonctionnement du programme, cette classe devra toujours être active.\\

- Communication client-serveur :

Le programme client et le serveur communiquent entre eux à travers des tubes de communications privés et publics. Les messages envoyés entre eux sont codifiés et limités.
Lorsqu'une session utilisateur est lancée, il notifie le serveur à travers un tube public en envoyant son identifiant (Process ID). 
Dés lors, le serveur lui crée des tubes privés pour conserver l'intégrité des données.\\

- Connexion :

Lorsqu'un client souhaite se connecter ou s'inscrire, il envoie son pseudo et mot de passe au serveur. Ce dernier se charge de vérifier l'existence ou la disponibilité de ces informations dans la base de données et ainsi ouvrir une session utilisateur ou non.\\

- Interactions entre utilisateurs :

Un client peut ajouter et supprimer des amis ainsi que voir le classement de tous les utilisateurs. 
Ces actions sont rendues possibles grâce au serveur qui sert de lien entre tous les programmes clients. 
En effet, ces derniers envoient les messages correspondants à leurs requêtes et le serveur se charge d'interroger la base de donnée et d'envoyer les réponses adéquates. \\

- Jeu :

Toutes les parties lancées par les utilisateurs sont gérées par le serveur. Ce qui signifie que c'est lui qui invoque les ennemis, les obstacles et les bonus. Le serveur reçoit également les mouvements des joueurs, les applique et vérifie les collisions.
Dés lors, le seul rôle du programme utilisateur est d'attendre les instructions du serveur et de les appliquer. 

Une fois qu'une partie est terminée, le serveur se charge de mettre à jour le score des joueurs. Le score final est commun et est la somme des scores de chaque joueur.

- Déconnexion du jeu :

Lorsqu'une session cliente est terminée, le serveur est mis au courant et se charge donc de supprimer les tubes de communications qui lui sont liés. S'il y a une partie en cours, elle est arrêtée. 
Dans le cas où le serveur est arrêté, tous les programmes utilisateurs ainsi que les parties lancées sont stoppées et la base de données est sauvegardée. \\
\begin{figure}
\centering
\includegraphics[scale=0.21]{images/ingameGui.png}
\caption{Jeu version graphique}

\includegraphics[scale=0.34]{images/ingameTerm.png}
\caption{Jeu version terminale}
\end{figure}

\begin{figure}
\centering
\includegraphics[scale=0.3]{images/newGameClassDiagram.jpg}
\caption{Diagramme de classe du jeu}
\end{figure}

\begin{figure}
\centering
\includegraphics[scale=0.3]{images/mapHandlerDiagram.jpg}
\caption{classe MapHandler}
\end{figure}
\newpage

\subsubsection{DataBase}

La base de données a pour rôle de sauvegarder les informations relatives aux utilisateurs. Un certains nombre d'opérations lui sont applicable à travers le serveur.
Les informations contenues dans la base de données sont les pseudonymes, mots de passe et le meilleur score de chaque utilisateur. Au final, celle-ci contient les niveaux créés par les utilisateurs, le nombre de votes qui sont liés à ces niveaux ainsi que la liste d'amis et de demandes d'amis pour chaque compte. 

\subsubsection{Client}
La classe Client envoie ses requêtes au serveur grâce aux menus. Toutes les fonctionnalités décrites dans la section 2.1 sont possibles. A savoir : 

\begin{itemize}
    \item Se connecter
    \item S'inscrire
    \item Créer une partie
    \item Consulter le classement des joueurs
    \item Gérer ses amis
    \item Consulter son profil
    \item Créer un niveau personnalisé
    \item Jouer à ses niveaux
    \item Jouer aux niveaux créés par les autres joueurs
    \item Voter plusieurs fois pour ses niveaux favoris
    \item Consulter le classement des niveaux\\\\
\end{itemize}
Le rôle principal du client consiste en l'affichage du jeu pour l'utilisateur. Tout au long de la partie, il se chargera de l'aspect audiovisuel (musique, effets sonores, arrière-plan, etc). 

\begin{figure}
\centering
\includegraphics[scale=0.3, angle=90]{images/newClientClassDiagram.jpg}
\caption{Diagramme de classe côté client version terminal}
\end{figure}
\begin{figure}
\centering
\includegraphics[scale=0.3, angle=90]{images/newClientGUIClassDiagram.png}
\caption{Diagramme de classe côté client version Gui}
\end{figure}
\newpage


\subsubsection{Gestion des comptes}
Un utilisateur est représenté  par un objet AccountAndVectors
qui contient sa liste d'amis, ses demandes d'amis , les niveaux qu'il a créé et ses informations personnelles (pseudo, mot de passe, score). Tous ces éléments sont stockés dans la base de données.\\

- Accès:

Lors de la création d'un compte, la disponibilité du pseudo est vérifiée par la base de données. Si celui-ci n'est pas trouvé, un nouveau compte est bien créé et ajouté dans la base.

Lors de la connexion, c'est encore le serveur, à travers la base de données qui vérifie que le pseudo et le mot de passe saisis correspondent à un compte existant.\\

- Contenu d'un compte:

Chaque compte possède des informations sur l'utilisateur auquel il appartient, notamment ses identifiants (pseudo, mot de passe), son score (pour qu'il apparaisse dans le classement général des joueurs), une liste d'amis, une liste d'invitations qu'il peut consulter à tout moment ainsi qu'une liste de ses niveaux créés et leurs nombre de votes.

\subsubsection{Gestion d'une partie}

Lorsque le joueur voudra lancer une partie,
il aura la possibilité de choisir les paramètres de celle-ci(cf 2.1.3). Si le nombre de joueurs choisi est 2, l'hôte doit obligatoirement inviter un autre utilisateur à se connecter.

La partie ne peut pas commencer tant que le second joueur ne réussit pas à se connecter.
\subsubsection{Gestion des amis}

- Ajout: 
Lorsque l'utilisateur veut ajouter un ami, la requête est envoyée à la base de données via le serveur. Après les vérifications, si la demande est valide, elle est envoyée à la personne concernée. Si celle-ci est acceptée, la liste d'amis des deux utilisateurs est mise à jour.

- Suppression: 
La base de données fait une recherche de l'ami à supprimer et l'efface de la liste d'amis de l'utilisateur. L'utilisateur sera également supprimé de la liste de son ancien ami.

\subsubsection{Classement}

Après chaque partie, le serveur met à jour le score de chaque joueur si celui-ci est supérieur à son score actuel. 
Ce qui permet de garder le classement tout le temps à jour. 

\begin{figure}
\centering
\includegraphics[scale=0.5]{images/add_friend.jpg}
\caption{Diagramme de séquence pour l'ajout d'un ami }
\end{figure}

\newpage
\begin{figure}
\centering
\includegraphics[scale=0.5]{images/del_friend.jpg}
\caption{Diagramme de séquence pour la suppression d'un ami }
\end{figure}

\subsection{Besoins non fonctionnels}
\begin{itemize}
    \item Le lancement du programme nécessite un environnement Linux possédant un dossier nommé "tmp" à la racine, afin de pouvoir stocker et supprimer les tubes créés par le serveur
    \item Par souci de sécurité, toutes les requêtes d'un client doivent passer par le serveur
    \item Pour des raisons esthétiques, la taille du terminal doit être de 80x24
    \item La librairie open source "ncurses" est indispensable pour la version terminale
    \item La librairie open source "Qt" est indispensable pour la version graphique
    \item La librairie open source "SFML" est indispensable pour la version graphique
    
\end{itemize}

\begin{figure}
\centering
\includegraphics[scale=0.3]{images/sequence_diagram_StartGame}
\caption{Diagramme de séquence de lancement du jeu}
\end{figure}

\begin{figure}
\centering
\includegraphics[scale=0.3]{images/player_shooting.jpg}
\caption{Diagramme de séquence d'un tir de joueur}
\end{figure}


\begin{figure}[hbtp]
\centering
\includegraphics[scale=0.3]{images/enemy_shooting.jpg}
\caption{Diagramme de séquence d'un tir ennemi }
\end{figure}

\newpage
\appendix
\section{ Annexe: Description du diagramme use case utilisateur}

\begin{center}
\begin{longtable}{|p{1,5cm}||p{3,5cm}|p{3,5cm}|p{3,5cm}|p{3,5cm}|}
\hline
\rowcolor{green}
USE CASE   &\center{Pré-conditions}   & \hfill Post-conditions \hfill\null & Cas Général & Cas exceptionnels\\
\hline
\hline
\textbf{Sign in}      & L'utilisateur doit être enregistré dans la base de données  & L'utilisateur est connecté à sa base de données et le menu principal est affiché & L'utilisateur déjà enregistré se connecte à son compte en entrant son pseudonyme et mot de passe. Le serveur vérifie que les données soient correctes et donne accès au compte du client  & Si l’identifiant ou le mot de passe sont incorrects, affiche un message d'erreur est affiché à l'utilisateur \\
\hline
\hline
\textbf{Sign up}     & L'utilisateur n'est pas présent dans la base de données   & Ajout d'un compte dans la base de données et affichage du menu principal & L'utilisateur crée un compte en introduisant un pseudo et un mot de passe  & Si les données entrées ne respectent pas le format requis ou que le pseudonyme est déjà utilisé, un message d'erreur est affiché et l'utilisateur peut recommencer l'action jusqu'à ce qu'elle soit valide \\

\hline
\hline
\textbf{Create game}    & L'utilisateur doit être enregistré dans la base de données  & Possibilité de sauvegarder les paramètres par défaut d'une partie  & L'utilisateur peut lancer une partie après avoir rempli les conditions minimales  & Néant \\
\hline
\hline
\textbf{Check Leaderboard}  & L'utilisateur doit être enregistré dans la base de données   & \hfill Néant  \hfill\null &Le joueur peut consulter le classement des scores en envoyant une requête au serveur qui va lui renvoyer les informations  & Néant \\
\hline
\hline
\textbf{View friend list }   & L'utilisateur doit être enregistré dans la base de données   & Néant  & Consultation de liste d'ami dans la base de données & Néant \\
\hline
\hline
\textbf{Add friend}    & L'utilisateur doit être enregistré dans la base de données   & Si invitation acceptée, ajout d'amis dans la base de donnée (bidirectionnel) & Entrer le pseudo d’un utilisateur. Le système va rechercher dans la base de données si le pseudo existe et lui envoyer une invitation.  & Ajouter un pseudo qui n’existe pas (affiche une erreur)\\
\hline
\hline
\textbf{Delete friend}    & L'utilisateur doit être enregistré dans la base de données et avoir au moins un ami.   & Suppression dans la liste d’amis par le serveur(bidirectionnel)  & Entrer le pseudo d’un ami. Le serveur va rechercher dans la base de données si le pseudo existe et le supprimer.  & Supprimer un ami qui n’existe pas (affiche une erreur) \\
\hline
\hline
\textbf{Profile}     & L'utilisateur doit être enregistré dans la base de données   & Mise à jour des changements  & L'utilisateur peut consulter ses informations de profile. Neant  & Néant\\
\hline
\hline
\textbf{Ship’s controls}  & Créer une partie  & Actualisation de l’état de jeu  & Se déplacer, tirer recevoir des bonus  & Néant\\
\hline
\textbf{Leave party}      & Être en train de jouer   & Retour au menu principal  & Le joueur arrête la partie en cours.  & Néant\\
\hline
\hline
\hline
\hline
\textbf{Leave game}      & Etre connecté   & Fermeture du jeu  & L'utilisateur est connecté et veut quitter le jeu  & L'utilisateur est connecté et force sa sortie du jeu (CTRL+C, ...)\\
\hline
\hline
\textbf{Create Level}      & Etre connecté sur la version gui  &L'ouverture de l'éditeur & L'utilisateur est connecté et veut crée un niveau  & Neant\\
\hline
\hline
\textbf{View Level}      & Etre connecté sur la version gui  &Classement de tous les niveaux ou classement de ses niveaux &L'utilisateur peut consulter les niveaux, voter pour n'importe lequel et jouer un niveau& Neant\\
\hline
\end{longtable}
\end{center}


\end{document}
